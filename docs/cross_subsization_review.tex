% Options for packages loaded elsewhere
% Options for packages loaded elsewhere
\PassOptionsToPackage{unicode}{hyperref}
\PassOptionsToPackage{hyphens}{url}
\PassOptionsToPackage{dvipsnames,svgnames,x11names}{xcolor}
%
\documentclass[
  11pt,
]{article}
\usepackage{xcolor}
\usepackage[margin=1in]{geometry}
\usepackage{amsmath,amssymb}
\setcounter{secnumdepth}{-\maxdimen} % remove section numbering
\usepackage{iftex}
\ifPDFTeX
  \usepackage[T1]{fontenc}
  \usepackage[utf8]{inputenc}
  \usepackage{textcomp} % provide euro and other symbols
\else % if luatex or xetex
  \usepackage{unicode-math} % this also loads fontspec
  \defaultfontfeatures{Scale=MatchLowercase}
  \defaultfontfeatures[\rmfamily]{Ligatures=TeX,Scale=1}
\fi
\usepackage{lmodern}
\ifPDFTeX\else
  % xetex/luatex font selection
\fi
% Use upquote if available, for straight quotes in verbatim environments
\IfFileExists{upquote.sty}{\usepackage{upquote}}{}
\IfFileExists{microtype.sty}{% use microtype if available
  \usepackage[]{microtype}
  \UseMicrotypeSet[protrusion]{basicmath} % disable protrusion for tt fonts
}{}
\usepackage{setspace}
\makeatletter
\@ifundefined{KOMAClassName}{% if non-KOMA class
  \IfFileExists{parskip.sty}{%
    \usepackage{parskip}
  }{% else
    \setlength{\parindent}{0pt}
    \setlength{\parskip}{6pt plus 2pt minus 1pt}}
}{% if KOMA class
  \KOMAoptions{parskip=half}}
\makeatother
% Make \paragraph and \subparagraph free-standing
\makeatletter
\ifx\paragraph\undefined\else
  \let\oldparagraph\paragraph
  \renewcommand{\paragraph}{
    \@ifstar
      \xxxParagraphStar
      \xxxParagraphNoStar
  }
  \newcommand{\xxxParagraphStar}[1]{\oldparagraph*{#1}\mbox{}}
  \newcommand{\xxxParagraphNoStar}[1]{\oldparagraph{#1}\mbox{}}
\fi
\ifx\subparagraph\undefined\else
  \let\oldsubparagraph\subparagraph
  \renewcommand{\subparagraph}{
    \@ifstar
      \xxxSubParagraphStar
      \xxxSubParagraphNoStar
  }
  \newcommand{\xxxSubParagraphStar}[1]{\oldsubparagraph*{#1}\mbox{}}
  \newcommand{\xxxSubParagraphNoStar}[1]{\oldsubparagraph{#1}\mbox{}}
\fi
\makeatother


\usepackage{longtable,booktabs,array}
\usepackage{calc} % for calculating minipage widths
% Correct order of tables after \paragraph or \subparagraph
\usepackage{etoolbox}
\makeatletter
\patchcmd\longtable{\par}{\if@noskipsec\mbox{}\fi\par}{}{}
\makeatother
% Allow footnotes in longtable head/foot
\IfFileExists{footnotehyper.sty}{\usepackage{footnotehyper}}{\usepackage{footnote}}
\makesavenoteenv{longtable}
\usepackage{graphicx}
\makeatletter
\newsavebox\pandoc@box
\newcommand*\pandocbounded[1]{% scales image to fit in text height/width
  \sbox\pandoc@box{#1}%
  \Gscale@div\@tempa{\textheight}{\dimexpr\ht\pandoc@box+\dp\pandoc@box\relax}%
  \Gscale@div\@tempb{\linewidth}{\wd\pandoc@box}%
  \ifdim\@tempb\p@<\@tempa\p@\let\@tempa\@tempb\fi% select the smaller of both
  \ifdim\@tempa\p@<\p@\scalebox{\@tempa}{\usebox\pandoc@box}%
  \else\usebox{\pandoc@box}%
  \fi%
}
% Set default figure placement to htbp
\def\fps@figure{htbp}
\makeatother


% definitions for citeproc citations
\NewDocumentCommand\citeproctext{}{}
\NewDocumentCommand\citeproc{mm}{%
  \begingroup\def\citeproctext{#2}\cite{#1}\endgroup}
\makeatletter
 % allow citations to break across lines
 \let\@cite@ofmt\@firstofone
 % avoid brackets around text for \cite:
 \def\@biblabel#1{}
 \def\@cite#1#2{{#1\if@tempswa , #2\fi}}
\makeatother
\newlength{\cslhangindent}
\setlength{\cslhangindent}{1.5em}
\newlength{\csllabelwidth}
\setlength{\csllabelwidth}{3em}
\newenvironment{CSLReferences}[2] % #1 hanging-indent, #2 entry-spacing
 {\begin{list}{}{%
  \setlength{\itemindent}{0pt}
  \setlength{\leftmargin}{0pt}
  \setlength{\parsep}{0pt}
  % turn on hanging indent if param 1 is 1
  \ifodd #1
   \setlength{\leftmargin}{\cslhangindent}
   \setlength{\itemindent}{-1\cslhangindent}
  \fi
  % set entry spacing
  \setlength{\itemsep}{#2\baselineskip}}}
 {\end{list}}
\usepackage{calc}
\newcommand{\CSLBlock}[1]{\hfill\break\parbox[t]{\linewidth}{\strut\ignorespaces#1\strut}}
\newcommand{\CSLLeftMargin}[1]{\parbox[t]{\csllabelwidth}{\strut#1\strut}}
\newcommand{\CSLRightInline}[1]{\parbox[t]{\linewidth - \csllabelwidth}{\strut#1\strut}}
\newcommand{\CSLIndent}[1]{\hspace{\cslhangindent}#1}



\setlength{\emergencystretch}{3em} % prevent overfull lines

\providecommand{\tightlist}{%
  \setlength{\itemsep}{0pt}\setlength{\parskip}{0pt}}






\makeatletter
\@ifpackageloaded{caption}{}{\usepackage{caption}}
\AtBeginDocument{%
\ifdefined\contentsname
  \renewcommand*\contentsname{Table of contents}
\else
  \newcommand\contentsname{Table of contents}
\fi
\ifdefined\listfigurename
  \renewcommand*\listfigurename{List of Figures}
\else
  \newcommand\listfigurename{List of Figures}
\fi
\ifdefined\listtablename
  \renewcommand*\listtablename{List of Tables}
\else
  \newcommand\listtablename{List of Tables}
\fi
\ifdefined\figurename
  \renewcommand*\figurename{Figure}
\else
  \newcommand\figurename{Figure}
\fi
\ifdefined\tablename
  \renewcommand*\tablename{Table}
\else
  \newcommand\tablename{Table}
\fi
}
\@ifpackageloaded{float}{}{\usepackage{float}}
\floatstyle{ruled}
\@ifundefined{c@chapter}{\newfloat{codelisting}{h}{lop}}{\newfloat{codelisting}{h}{lop}[chapter]}
\floatname{codelisting}{Listing}
\newcommand*\listoflistings{\listof{codelisting}{List of Listings}}
\makeatother
\makeatletter
\makeatother
\makeatletter
\@ifpackageloaded{caption}{}{\usepackage{caption}}
\@ifpackageloaded{subcaption}{}{\usepackage{subcaption}}
\makeatother
\usepackage{bookmark}
\IfFileExists{xurl.sty}{\usepackage{xurl}}{} % add URL line breaks if available
\urlstyle{same}
\hypersetup{
  pdftitle={Cross-Subsidization in Electric Utility Rate Design: Analytical Frameworks and Practical Considerations},
  colorlinks=true,
  linkcolor={blue},
  filecolor={Maroon},
  citecolor={Blue},
  urlcolor={Blue},
  pdfcreator={LaTeX via pandoc}}


\title{Cross-Subsidization in Electric Utility Rate Design: Analytical
Frameworks and Practical Considerations}
\author{}
\date{2025-08-06}
\begin{document}
\maketitle


\setstretch{2}
\subsection{Introduction}\label{introduction}

Cross-subsidization in electric utility rate design constitutes a core
challenge in regulatory economics, involving the systematic transfer of
costs between customer classes through rate structures that deviate from
strict cost-causation principles. The characterization and measurement
of cross-subsidization has evolved significantly from the foundational
works of Bonbright (1961) and Kahn (1971) through contemporary
frameworks incorporating advanced metering infrastructure and
distributed energy resources. This document provides an analysis of
cross-subsidization concepts, examining the formal definitions and
analytical methods developed in the utility economics literature, with
particular focus on the frameworks established by Lazar and Gonzalez
(2020) in their treatment of cost allocation methods and the Bill
Alignment Test methodology introduced by Simeone et al. (2023).

The central thesis emerging from this literature is that
cross-subsidization measurement requires explicit mathematical
formalization of policy preferences, rather than relying solely on
technical cost allocation procedures. This evolution from positive to
normative analytical frameworks reflects both the technological
capabilities enabled by smart metering infrastructure and the growing
recognition that utility rate design inherently involves policy
trade-offs that cannot be resolved through purely economic optimization.
As Simeone et al. (2023) demonstrate through their analysis of over
35,000 hourly customer load profiles, the choice of residual cost
allocation method---which embodies regulatory policy preferences---often
has greater impact on cross-subsidization patterns than the choice
between flat and time-varying rate structures.

Understanding cross-subsidization requires appreciation of several
fundamental developments in utility regulation: (1) the sequential
nature of rate-making processes, (2) the role of cost drivers in
determining appropriate allocation methods, (3) the distinct purposes
served by embedded versus marginal cost frameworks, (4) how
technological change challenges traditional allocation methods, and (5)
how modern analytical tools bridge cost allocation theory with customer
impact assessment. These themes structure our following discussion.

\subsection{The Sequential Structure of Utility
Rate-Making}\label{the-sequential-structure-of-utility-rate-making}

\subsubsection{The Three-Step Process and Its
Implications}\label{the-three-step-process-and-its-implications}

Modern utility regulation, as described by Lazar and Gonzalez (2020),
involves a three-step process in which ``each phase feeds into the
next'' and ``the analysis is inevitably sequential,'' meaning that
decisions made in early phases create constraints that shape options in
later stages.

The first step, \textbf{revenue requirement determination}, establishes
the total annual revenue \(R\) that the utility must collect to both
recover prudently incurred operating expenses and earn a fair return on
capital investments used and useful in providing service. This phase
typically receives the most regulatory attention, involving detailed
examination of utility expenditures, rate base valuation, depreciation
schedules, and the allowed rate of return (Joskow 2007). The revenue
requirement can be expressed mathematically as:

\begin{equation}\phantomsection\label{eq-revenue}{R = O + D + T + (RB \times ROR)}\end{equation}

where \(O\) represents operating expenses, \(D\) is depreciation, \(T\)
represents taxes, \(RB\) is the rate base (net plant in service plus
working capital), and \(ROR\) is the allowed rate of return.

The second step, \textbf{cost allocation}, divides this revenue
requirement among customer classes. Here, we determine how much each
customer class---typically residential, commercial, and industrial,
though classifications vary by jurisdiction---should contribute toward
total revenue recovery. The allocation process theoretically follows
cost-causation principles, attempting to assign costs to the customers
who cause them to be incurred. However, as Lazar and Gonzalez (2020)
document, this process involves substantial policy judgments about
appropriate allocation methods, the treatment of joint and common costs,
and considerations of fairness and equity that go beyond technical cost
accounting. Formally, we require that:

\begin{equation}\phantomsection\label{eq-allocation}{R = \sum_{i=1}^{n} R_i}\end{equation}

where \(R_i\) represents the revenue allocated to customer class \(i\).

The third step, \textbf{rate design}, translates class-level revenue
allocations into specific tariff structures that individual customers
within each class will face. While rate design principles emphasize
forward-looking economic efficiency and the provision of appropriate
price signals to influence customer behavior (Borenstein 2005), rate
designers operate within binding constraints established by the cost
allocation phase. The designed tariffs must satisfy the revenue
requirement for each class:

\begin{equation}\phantomsection\label{eq-tariff}{\sum_{j \in i} B_j(T_i) = R_i}\end{equation}

where \(B_j(T_i)\) represents the annual bill for customer \(j\) under
tariff \(T_i\) designed for class \(i\).

This sequential nature creates path dependence in regulatory
decision-making. Rate designers in step three face binding constraints
on how much revenue each class must contribute even if optimal rate
design principles suggest otherwise. As a result, cross-subsidization
patterns embedded during the cost allocation phase cannot be fully
corrected through rate design alone. Institutional inertia develops as
cross-subsidizing patterns established in one regulatory proceeding
persist through multiple rate cycles due to the procedurally complex and
politically contentious nature of reopening cost allocation.

The Bill Alignment Test methodology developed by Simeone et al. (2023)
addresses this by enabling ex-post evaluation of whether final rate
outcomes align with stated regulatory preferences. By comparing customer
bills to allocated costs under explicit policy preferences for residual
cost treatment, the BAT can identify when the accumulation of sequential
decisions has produced outcomes inconsistent with regulatory intent.

\subsection{Cost Drivers and the Challenge of Cost
Causation}\label{cost-drivers-and-the-challenge-of-cost-causation}

\subsubsection{Understanding Cost Causation in Complex Electric
Systems}\label{understanding-cost-causation-in-complex-electric-systems}

Effective cost allocation and cross-subsidization analysis depend
fundamentally on identifying what Lazar and Gonzalez (2020) term ``cost
drivers'', which can be complex in electricity systems for various
reasons. For one, electricity cannot be stored economically at scale
(though this is changing with battery technology), requiring real-time
balancing of supply and demand. Additionally, the shared network
infrastructure means that one customer's usage decisions affect system
costs experienced by all customers. Finally, energy provisioning
involves clearing the market not just in the spot electricity market,
but also in the ancillary services and capacity markets (Joskow 2019).

\subsubsection{Traditional Cost Categories and Their
Limitations}\label{traditional-cost-categories-and-their-limitations}

Traditional utility regulation has relied on broad categorizations of
costs as customer-related, demand-related, or energy-related, a
framework formalized in the National Association of Regulatory Utility
Commissioners (1992) cost allocation manual that has guided regulatory
practice for decades.

However, the evolution of modern electricity systems challenges these
traditional categories. For example, advanced metering infrastructure
has traditionally been classified as a customer-related cost since
meters primarily served billing functions. Yet modern smart meters
provide system-wide benefits including voltage control, outage
management, demand response enablement, and distributed resource
integration that extend far beyond individual customer service (Hledik
2014). Similarly, renewable energy resources challenge traditional
demand/energy classifications. As Lazar and Gonzalez (2020) observe,
``wind and solar generation does not necessarily provide firm capacity
at peak times as envisioned by the legacy frameworks, and it displaces
the need for fuel supply, so it doesn't fit as a demand-related cost.''
These resources provide energy value when available but may contribute
little to system reliability during peak periods, requiring new
analytical approaches that can accommodate their unique cost and value
characteristics.

\subsubsection{Load Diversity and System Cost
Implications}\label{load-diversity-and-system-cost-implications}

One of the fundamental insights in utility cost allocation is the role
of load diversity, which Lazar and Gonzalez (2020) define mathematically
as the difference between the sum of individual customer peak loads
(non-coincident peak or NCP) and the actual system peak load (coincident
peak or CP). They explain that ``the shared portions of the system need
to be sized to meet only the coincident peak loads for combined customer
usage at each point of the system, rather than the sum of the customers'
noncoincident peak loads.'' This diversity benefit arises because
customers reach their maximum demand at different times, allowing shared
infrastructure to be smaller than would be required if all customers
peaked simultaneously.

The mathematical representation of load diversity is:

\begin{equation}\phantomsection\label{eq-diversity}{LD = \sum_j \max_h(q_{jh}) - \max_h\left(\sum_j q_{jh}\right)}\end{equation}

where \(q_{jh}\) represents customer \(j\)'s load in hour \(h\). The
diversity factor, defined as the ratio of coincident to non-coincident
peak, typically ranges from 0.3 to 0.7 for residential customers,
meaning that shared infrastructure needs only 30-70\% of the capacity
that would be required without diversity benefits.

The allocation of diversity benefits among customers significantly
affects cross-subsidization patterns. Customers whose individual peak
usage occurs at times different from the system peak impose lower costs
on shared infrastructure, while those whose peaks coincide with system
peaks drive capacity requirements throughout the generation,
transmission, and distribution systems. Yet traditional allocation
methods may not fully capture these timing differences, potentially
creating cross-subsidization between customers with different load
profiles even within the same rate class.

\subsection{Theoretical Foundations and Mathematical
Definitions}\label{theoretical-foundations-and-mathematical-definitions}

\subsubsection{Evolution from Game-Theoretic to Regulatory
Frameworks}\label{evolution-from-game-theoretic-to-regulatory-frameworks}

The mathematical foundation for cross-subsidization analysis emerged
from the intersection of cooperative game theory and public utility
regulation in the 1970s. Faulhaber (1975) provided the first rigorous
definition using game-theoretic sustainability concepts, establishing
mathematical bounds that remain influential in contemporary regulatory
analysis. However, the evolution from these theoretical foundations to
practical regulatory frameworks reveals fundamental tensions between
economic theory and regulatory implementation that continue to shape
modern cross-subsidization analysis.

Faulhaber's framework treats utility service as a cooperative game where
customer classes must agree to share costs in a manner that prevents any
subset from preferring to ``go it alone'' by self-providing service or
seeking alternative suppliers. For a utility serving customer classes
\(i \in \{1, 2, ..., n\}\), let \(C(S)\) represent the stand-alone cost
of serving customer subset \(S \subseteq N\), where
\(N = \{1, 2, ..., n\}\) is the set of all customer classes. The
incremental cost of serving class \(i\) is defined as:

\begin{equation}\phantomsection\label{eq-incremental}{IC_i = C(N) - C(N \setminus \{i\})}\end{equation}

This represents the additional cost imposed on the system by serving
class \(i\). The stand-alone cost for class \(i\) is simply
\(SAC_i = C(\{i\})\), the cost of serving that class in isolation.
Faulhaber's sustainability condition requires that prices satisfy:

\begin{equation}\phantomsection\label{eq-sustainability}{IC_i \leq P_i \leq SAC_i \quad \forall i \in N}\end{equation}

This framework establishes mathematical bounds for subsidy-free pricing:
prices below incremental cost constitute cross-subsidization of class
\(i\) by other classes (since class \(i\) is not covering the costs it
imposes), while prices above stand-alone cost represent
cross-subsidization by class \(i\) to others (since class \(i\) could
obtain service more cheaply independently).

The limitation of this approach, in addition to requiring a detailed
estimate of \(IC_i\) and \(SAC_i\), is that the bounds are often quite
wide. Stand-alone costs may be several times incremental costs,
providing little guidance for selecting specific prices within the
subsidy-free range.

\subsubsection{The Marginal Cost Alternative and Efficiency
Foundations}\label{the-marginal-cost-alternative-and-efficiency-foundations}

Kahn (1971) established an alternative foundation for
cross-subsidization analysis through marginal cost pricing theory. This
approach offers a way to select prices based on economic welfare
maximization, providing a standard for optimal pricing.

In the utility context, marginal cost \(MC_i(q_i)\) represents the cost
of serving an additional unit of consumption by customer class \(i\).
Under this framework, cross-subsidization occurs whenever prices deviate
from marginal cost:

\begin{equation}\phantomsection\label{eq-marginal-cs}{CS_i = P_i - MC_i}\end{equation}

where \(CS_i > 0\) indicates that class \(i\) is paying above marginal
cost and therefore subsidizing others, while \(CS_i < 0\) indicates that
class \(i\) is paying below marginal cost and receiving subsidies.

The connection between marginal cost deviations and economic efficiency
operates through standard welfare analysis. When prices exceed marginal
cost, consumers reduce consumption below economically efficient levels,
creating deadweight loss from foregone consumption that would have
generated benefits exceeding costs. When prices fall below marginal
cost, consumers increase consumption beyond efficient levels, consuming
units where costs exceed benefits. The deadweight loss from these price
distortions can be approximated as:

\begin{equation}\phantomsection\label{eq-dwl}{DWL_i = \frac{1}{2} \epsilon_i \frac{(P_i - MC_i)^2}{P_i}}\end{equation}

where \(\epsilon_i\) is the price elasticity of demand for class \(i\).
Thus, marginal cost pricing serves both as a cross-subsidization test
and an efficiency standard, with deviations creating both distributional
and efficiency consequences.

\subsubsection{Second-Best Theory and the Revenue Adequacy
Problem}\label{second-best-theory-and-the-revenue-adequacy-problem}

The practical application of marginal cost pricing in electricity
markets encounters what is known as the revenue adequacy problem or
``missing money'' problem. Electric utilities are characterized by high
fixed costs for generation, transmission, and distribution
infrastructure combined with relatively low (sometimes near-zero)
short-run marginal costs, particularly in systems with substantial
renewable generation. Pricing at short-run marginal cost would fail to
recover total costs, threatening financial viability.

This revenue insufficiency challenge led to the development of
second-best pricing theory, formalized through Ramsey pricing
principles. Baumol, Panzar, and Willig (1982) extended the marginal cost
framework for multi-product utilities with joint and common costs that
cannot be attributed to specific services. The optimal second-best
pricing rule minimizes deadweight loss subject to a revenue adequacy
constraint:

\begin{equation}\phantomsection\label{eq-ramsey}{\frac{P_i - MC_i}{P_i} = \frac{\lambda}{1 + \lambda} \cdot \frac{1}{\epsilon_i}}\end{equation}

where \(\lambda\) is the Lagrange multiplier on the revenue constraint
(representing the shadow cost of the revenue requirement) and
\(\epsilon_i\) is the price elasticity of demand for class \(i\).

Optimal second-best pricing necessarily involves cross-subsidization in
the marginal cost sense, since prices must deviate from marginal cost to
achieve revenue adequacy. However, these deviations follow systematic
patterns based on demand elasticities rather than arbitrary regulatory
preferences. Customer classes with inelastic demand (low \(\epsilon_i\))
optimally bear larger markups over marginal cost, effectively
subsidizing classes with elastic demand. This is economically efficient
because it minimizes the distortion in consumption patterns required to
meet the revenue constraint.

\subsection{Cost Allocation Framework: From Theory to
Implementation}\label{cost-allocation-framework-from-theory-to-implementation}

\subsubsection{Embedded Cost Allocation}\label{embedded-cost-allocation}

Lazar and Gonzalez (2020) provide a characterization of embedded cost
allocation, aligning with the three-step framework discussed earlier.
The embedded cost framework begins with the fundamental requirement that
all costs must be allocated. Let \(R\) represent the utility's total
revenue requirement, established through the rate-making process
described in \textbf{?@sec-sequential}. Let \(C_{f,c,a}\) represent
costs categorized by function \(f\), classification \(c\), and
allocation method \(a\).

\textbf{Step 1: Functionalization} assigns costs to system functions
based on the primary purpose served by each asset or expense category.
Lazar and Gonzalez (2020) identify the major functions as generation
\((G)\), transmission \((T)\), distribution \((D)\), and customer
service \((C)\). The functionalization requirement is:

\begin{equation}\phantomsection\label{eq-functionalization}{\sum_{f \in \{G,T,D,C\}} C_f = R}\end{equation}

This equation ensures that every dollar of revenue requirement must be
assigned to some system function. However, the assignment process
involves regulatory judgment, especially for shared assets that serve
multiple functions simultaneously.

\textbf{Step 2: Classification} assigns functional costs to causation
categories based on the primary drivers of cost incurrence. The
traditional categories are energy-related, demand-related, and
customer-related:

\begin{equation}\phantomsection\label{eq-classification}{C_f = C_{f,E} + C_{f,D} + C_{f,C} \quad \forall f}\end{equation}

The energy category captures costs that vary with total electricity
consumption (primarily fuel and variable O\&M), the demand category
captures costs driven by peak usage requirements (capacity-related
infrastructure), and the customer category captures costs that vary with
the number of customers served regardless of usage levels (meters,
billing, service drops).

\textbf{Step 3: Allocation} distributes classified costs among customer
classes using allocation factors that reflect each class's contribution
to the relevant cost driver:

\begin{equation}\phantomsection\label{eq-final-allocation}{C_i = \sum_f \sum_c \alpha_{f,c,i} \cdot C_{f,c}}\end{equation}

where \(\alpha_{f,c,i}\) represents class \(i\)'s allocation factor for
costs in function \(f\) and classification \(c\), with the requirement
that \(\sum_i \alpha_{f,c,i} = 1\) for all \((f,c)\) combinations.

\subsubsection{Marginal Cost of Service Studies: Operationalizing
Efficiency
Theory}\label{marginal-cost-of-service-studies-operationalizing-efficiency-theory}

The translation of Kahn's marginal cost pricing theory into operational
regulatory methodology required development of what became known as
``marginal cost of service studies.'' These studies represent a distinct
analytical framework that attempts to implement efficiency principles
within the constraints of utility regulation.

The mathematical formulation calculates class-specific marginal cost
revenue requirements by multiplying estimated marginal costs by relevant
usage determinants:

\begin{equation}\phantomsection\label{eq-mcrr-class}{MCRR_i = \sum_j MC_j \cdot Q_{ij}}\end{equation}

where \(MC_j\) represents the marginal cost for system component \(j\)
and \(Q_{ij}\) represents class \(i\)'s usage of component \(j\)
(measured in appropriate units such as kWh for energy components, kW for
capacity components, or customer counts for customer-related
components).

The system-wide marginal cost revenue requirement aggregates across all
customer classes:

\begin{equation}\phantomsection\label{eq-mcrr-total}{MCRR = \sum_i MCRR_i}\end{equation}

This calculation typically yields a revenue requirement that differs
substantially from the embedded cost revenue requirement \(R\), creating
what Lazar and Gonzalez (2020) identify as the fundamental
reconciliation challenge.

\subsection{The Bill Alignment Test}\label{the-bill-alignment-test}

\subsubsection{Conceptual Foundation and Normative
Redefinition}\label{conceptual-foundation-and-normative-redefinition}

Simeone et al. (2023) introduce a paradigmatic shift in
cross-subsidization analysis by explicitly defining cross-subsidization
as a normative concept rather than attempting to establish objective
technical criteria. Their Bill Alignment Test (BAT) framework
acknowledges that any determination of cross-subsidization inherently
involves policy judgments about how costs should be allocated among
customers.

The BAT framework begins by calculating each customer's economic cost
based on marginal cost principles. For customer \(j\) in class \(i\),
the economic cost is:

\begin{equation}\phantomsection\label{eq-economic-cost}{EC_j = \sum_h MC_h \cdot q_{jh}}\end{equation}

where \(MC_h\) is the system marginal cost in hour \(h\) and \(q_{jh}\)
is customer \(j\)'s consumption in hour \(h\). This calculation uses
actual customer load data from smart meters, capturing the full temporal
variation in both costs and consumption patterns.

The total system cost allocated to customer \(j\) then depends on the
stated residual allocation preference \(\psi\):

\begin{equation}\phantomsection\label{eq-system-cost}{SC_j = EC_j + RS_j(\psi)}\end{equation}

where \(RS_j(\psi)\) represents customer \(j\)'s share of residual costs
under policy preference \(\psi\). The residual costs are defined as
\(RC = R - \sum_j EC_j\), the difference between the total revenue
requirement and the sum of all customers' economic costs.

\subsubsection{Residual Cost Allocation
Methods}\label{residual-cost-allocation-methods}

Simeone et al. (2023) implement and compare three different residual
cost allocation methods:

\textbf{Flat (Per-Customer) Method:} This approach allocates residual
costs equally among all customers:

\begin{equation}\phantomsection\label{eq-flat-residual}{RS_j^{flat} = \frac{RC}{N}}\end{equation}

where \(N\) is the total number of customers.

\textbf{Volumetric (Per-kWh) Method:} This approach allocates residual
costs proportionally to total consumption:

\begin{equation}\phantomsection\label{eq-vol-residual}{RS_j^{vol} = RC \cdot \frac{\sum_h q_{jh}}{\sum_j \sum_h q_{jh}}}\end{equation}

\textbf{Volumetric Excluding Low-Income Method:} This approach exempts
designated customer groups from residual cost allocation:

\begin{equation}\phantomsection\label{eq-vol-ex-residual}{RS_j^{vol-ex} = \begin{cases}
0 & \text{if } j \in LI \\
RC \cdot \frac{\sum_h q_{jh}}{\sum_{k \notin LI} \sum_h q_{kh}} & \text{if } j \notin LI
\end{cases}}\end{equation}

where \(LI\) denotes the set of low-income customers.

\subsubsection{Bill Alignment Calculation and
Interpretation}\label{bill-alignment-calculation-and-interpretation}

The bill alignment for customer \(j\) under tariff \(T\) and residual
allocation policy \(\psi\) is calculated as:

\begin{equation}\phantomsection\label{eq-bill-alignment}{BA_j(T,\psi) = B_j(T) - SC_j(\psi)}\end{equation}

where \(B_j(T)\) is customer \(j\)'s annual bill under tariff \(T\) and
\(SC_j(\psi)\) is their allocated system cost under policy \(\psi\).

The interpretation of bill alignment values is straightforward:

\begin{itemize}
\tightlist
\item
  \(BA_j > 0 \implies\) customer \(j\) overpays (provides cross-subsidy)
\item
  \(BA_j < 0 \implies\) customer \(j\) underpays (receives
  cross-subsidy)\\
\item
  \(BA_j = 0 \implies\) no cross-subsidization
\end{itemize}

\subsubsection{Aggregate Cross-Subsidy
Metrics}\label{aggregate-cross-subsidy-metrics}

\textbf{Average Cross-Subsidy (ACS):} This metric measures the average
absolute deviation from ideal bill alignment:

\begin{equation}\phantomsection\label{eq-acs}{ACS = \frac{1}{N} \sum_j |BA_j|}\end{equation}

Lower ACS values indicate that customer bills more closely match
allocated costs, suggesting less cross-subsidization overall.

\textbf{Directional Cross-Subsidy between Groups:} For analyzing
transfers between customer groups:

\begin{equation}\phantomsection\label{eq-dcs}{DCS_{G_1 \to G_2} = \frac{1}{|G_1|}\sum_{j \in G_1} \max(BA_j, 0) = -\frac{1}{|G_2|}\sum_{j \in G_2} \min(BA_j, 0)}\end{equation}

\textbf{Deadweight Loss (DWL):} To capture efficiency implications:

\begin{equation}\phantomsection\label{eq-dwl-bat}{DWL = \sum_h \sum_j \frac{\epsilon}{2} \cdot q_{jh} \cdot \left(\frac{P_h - MC_h}{P_h}\right)^2}\end{equation}

\subsection{Policy Implications and Trade-off
Analysis}\label{policy-implications-and-trade-off-analysis}

\subsubsection{Multi-Objective Optimization
Framework}\label{multi-objective-optimization-framework}

The BAT framework enables formal treatment of regulatory trade-offs
through multi-objective optimization:

\begin{equation}\phantomsection\label{eq-optimization}{\begin{align}
\min_{\psi, T} \quad & \mathcal{L}(\psi, T) = w_1 \cdot ACS(\psi, T) + w_2 \cdot DWL(\psi, T) \\
& \quad + w_3 \cdot IGT_{solar}(\psi, T) + w_4 \cdot LI_{burden}(\psi, T) \\
\text{s.t.} \quad & \sum_j B_j(T) = R \quad \text{(revenue adequacy)} \\
& |BA_j(\psi, T)| \leq \tau_j \quad \forall j \in \text{Protected classes} \\
& P_{min} \leq P_h(T) \leq P_{max} \quad \forall h \quad \text{(rate bounds)}
\end{align}}\end{equation}

where the weights \(w_i\) reflect regulatory priorities for minimizing
total cross-subsidies, deadweight loss, inter-group transfers, and
burden on vulnerable customers.

\subsubsection{Implementation
Recommendations}\label{implementation-recommendations}

Based on their empirical analysis, Simeone et al. (2023) offer several
recommendations for regulatory practice:

\begin{enumerate}
\def\labelenumi{\arabic{enumi}.}
\item
  \textbf{Explicit Policy Statements:} Regulators should explicitly
  state preferences for residual cost allocation before rate design
  begins.
\item
  \textbf{Customer-Level Analysis:} When smart meter data is available,
  analyze customer-level impacts rather than relying solely on class
  averages.
\item
  \textbf{Sensitivity Testing:} Evaluate rate proposals under multiple
  residual allocation methods to understand the robustness of
  conclusions.
\item
  \textbf{Multiple Metrics:} Report multiple metrics (ACS, DWL,
  inter-group transfers) to illuminate trade-offs rather than focusing
  on a single criterion.
\item
  \textbf{Regular Reassessment:} As technology and usage patterns
  evolve, periodically reassess whether existing rates achieve stated
  objectives.
\end{enumerate}

\subsection{References}\label{references}

\{\#refs\}

\phantomsection\label{refs}
\begin{CSLReferences}{1}{0}
\bibitem[\citeproctext]{ref-baumol1982contestable}
Baumol, William J, John C Panzar, and Robert D Willig. 1982.
\emph{Contestable Markets and the Theory of Industry Structure}. New
York: Harcourt Brace Jovanovich.

\bibitem[\citeproctext]{ref-bonbright1961principles}
Bonbright, James C. 1961. \emph{Principles of Public Utility Rates}. New
York: Columbia University Press.

\bibitem[\citeproctext]{ref-borenstein2005time}
Borenstein, Severin. 2005. {``Time-Varying Retail Electricity Prices:
Theory and Practice.''} \emph{Electricity Deregulation: Choices and
Challenges}, 317--57.

\bibitem[\citeproctext]{ref-faulhaber1975cross}
Faulhaber, Gerald R. 1975. {``Cross-Subsidization: Pricing in Public
Enterprises.''} \emph{The American Economic Review} 65 (5): 966--77.

\bibitem[\citeproctext]{ref-hledik2014role}
Hledik, Ryan. 2014. {``The Role of Advanced Metering Infrastructure in
the Electricity Sector.''} \emph{The Electricity Journal} 27 (6): 3--10.

\bibitem[\citeproctext]{ref-joskow2007regulation}
Joskow, Paul L. 2007. \emph{Regulation of Natural Monopoly}. Cambridge,
MA: MIT Press.

\bibitem[\citeproctext]{ref-joskow2019challenges}
---------. 2019. {``Challenges for Wholesale Electricity Markets with
Intermittent Renewable Generation at Scale: The US Experience.''}
\emph{Oxford Review of Economic Policy} 35 (2): 291--331.

\bibitem[\citeproctext]{ref-kahn1971economics}
Kahn, Alfred E. 1971. \emph{The Economics of Regulation: Principles and
Institutions}. New York: John Wiley \& Sons.

\bibitem[\citeproctext]{ref-lazar2020electric}
Lazar, Jim, and Wilson Gonzalez. 2020. {``Electric Utility Cost
Allocation for a New Era: A Manual.''} Montpelier, VT: Regulatory
Assistance Project.
\url{https://www.raponline.org/knowledge-center/electric-utility-cost-allocation-for-a-new-era-a-manual/}.

\bibitem[\citeproctext]{ref-naruc1992electric}
National Association of Regulatory Utility Commissioners. 1992.
\emph{Electric Utility Cost Allocation Manual}. Washington, DC: National
Association of Regulatory Utility Commissioners.

\bibitem[\citeproctext]{ref-simeone2023bill}
Simeone, Christina E, Pieter Gagnon, Peter Cappers, and Andrew
Satchwell. 2023. {``The Bill Alignment Test: Identifying Trade-Offs with
Residential Rate Design Options.''} \emph{Utilities Policy} 82: 101539.
\url{https://doi.org/10.1016/j.jup.2023.101539}.

\end{CSLReferences}




\end{document}
